\documentclass{article}
\usepackage{amsmath}
\usepackage{amssymb}
\usepackage{algorithm}
\usepackage{algpseudocode}
\usepackage{graphicx}
\usepackage{hyperref}
\usepackage{tikz}
\usepackage[a4paper,margin=1in]{geometry}

\hypersetup{
    colorlinks=true,
    linkcolor=blue,
    filecolor=magenta,      
    urlcolor=cyan,
}

\title{Programming Assignment 1: Advanced Machine Learning}
\author{
    Advanced Machine Unlearning \\
    Tejas Sharma: 22B0909 \\
    Muskaan Jain: 22B1058 \\
    Ojas Maheshwari: 22B0965
}
\date{}

\begin{document}

\maketitle
\tableofcontents
\clearpage

\section{Introduction}
Given an undirected graph with cliques and their associated potentials, this assignment focuses on computing marginal probabilities and the Maximum A Posteriori (MAP) assignment using message-passing algorithms. The input includes predefined cliques, which are complete subgraphs, ensuring structured factorization. The process involves triangulation, junction tree construction, and message-passing algorithms for inference.

\section{Triangulation}
\subsection{Overview}
Triangulation transforms an undirected graph into a chordal graph by adding edges such that every cycle of length greater than 3 has a chord. This ensures the graph can be decomposed into cliques, which is essential for constructing a junction tree.

\subsection{Steps}
\begin{enumerate}
    \item \textbf{Identify Simplicial Vertices}: A vertex is simplicial if its neighbors form a clique. Simplicial vertices are removed iteratively to determine the elimination order.
    \item \textbf{Elimination Order}: If no simplicial vertices are found, the vertex with the minimum degree is selected for elimination. If multiple simplicial vertices exist, they are eliminated simultaneously in arbitrary order.
    \item \textbf{Add Edges}: During elimination, edges are added between the neighbors of the eliminated vertex to maintain the chordal property.
    \item \textbf{Maximal Cliques}: The cliques formed during elimination are identified as maximal cliques.
\end{enumerate}

\subsection{Pseudocode}
\begin{algorithm}[H]
\caption{Triangulation and Maximal Clique Extraction}
\begin{algorithmic}[1]
\State Initialize adjacency list and empty list for maximal cliques.
\While{adjacency list is not empty}
    \State Find simplicial vertices.
    \If{simplicial vertices exist}
        \State Add simplicial vertices to the elimination order.
        \State Remove simplicial vertices and their edges.
        \State Add the clique formed by the simplicial vertex and its neighbors to maximal cliques.
    \Else
        \State Find the vertex with the minimum degree.
        \State Add missing edges between its neighbors to triangulate.
        \State Add the vertex and its neighbors to maximal cliques.
    \EndIf
\EndWhile
\State Remove redundant cliques.
\State Return maximal cliques.
\end{algorithmic}
\end{algorithm}

\section{Junction Tree Construction}
\subsection{Overview}
A junction tree is constructed from the maximal cliques of the triangulated graph. The tree must satisfy the running intersection property, ensuring that variables common to two cliques (separator variables) are present in all cliques on the path between them.

\subsection{Steps}
\begin{enumerate}
    \item \textbf{Create Nodes}: Each maximal clique becomes a node in the junction tree.
    \item \textbf{Connect Cliques}: Connect cliques that share common variables.
    \item \textbf{Assign Potentials}: Assign potentials to each node as the product of all potentials in the original graph whose variables are subsumed by that node.
\end{enumerate}

\subsection{Pseudocode}
\begin{algorithm}[H]
\caption{Junction Tree Construction}
\begin{algorithmic}[1]
\State Initialize an empty adjacency list for the junction tree.
\For{each pair of maximal cliques}
    \If{cliques share common variables}
        \State Connect the cliques in the junction tree.
    \EndIf
\EndFor
\State Assign potentials to cliques based on the original graph.
\State Return the junction tree.
\end{algorithmic}
\end{algorithm}

\section{Marginal Probability}
\subsection{Overview}
Marginal probabilities are computed using the sum-product algorithm on the junction tree. The algorithm propagates messages between cliques to compute the marginal distribution for each variable.

\subsection{Steps}
\begin{enumerate}
    \item \textbf{Initialize Potentials}: Assign initial potentials to the cliques.
    \item \textbf{Message Passing}: Propagate messages from leaf cliques to the root and back.
    \item \textbf{Marginalization}: Marginalize out all other variables to compute the final probabilities.
\end{enumerate}

\subsection{Pseudocode}
\begin{algorithm}[H]
\caption{Marginal Probability Computation}
\begin{algorithmic}[1]
\State Initialize all factors with potentials from the junction tree.
\For{each variable in elimination order}
    \State Collect factors containing the variable.
    \State Multiply factors and marginalize out the variable.
    \State Update the remaining factors.
\EndFor
\State Compute the partition function \( Z \).
\State Normalize the marginals using \( Z \).
\State Return the marginals.
\end{algorithmic}
\end{algorithm}

\section{MAP Assignment}
\subsection{Overview}
The MAP assignment is the most probable assignment of variables in the graphical model. It is computed using the max-product algorithm, which maximizes the joint probability distribution.

\subsection{Steps}
\begin{enumerate}
    \item \textbf{Initialize Potentials}: Assign initial potentials to the cliques.
    \item \textbf{Message Passing}: Propagate messages using the max-product rule.
    \item \textbf{Backtracking}: Determine the assignment that maximizes the joint probability.
\end{enumerate}

\section{Top \( k \) Assignments}
\subsection{Overview}
The top \( k \) assignments are the \( k \) most probable assignments of variables. These are computed by extending the max-product, message-passing algorithm to keep track of the top \( k \) configurations at each step.

\subsection{Steps}
\begin{enumerate}
    \item \textbf{Initialize Potentials}: Assign initial potentials to the cliques.
    \item \textbf{Message Passing}: Propagate messages while maintaining the top \( k \) assignments.
    \item \textbf{Normalization}: Normalize the probabilities using the partition function \( Z \).
\end{enumerate}

\subsection{Pseudocode}
\begin{algorithm}[H]
\caption{Top \( k \) Assignments}
\begin{algorithmic}[1]
\State Initialize all factors with potentials from the junction tree.
\For{each variable in elimination order}
    \State Collect factors containing the variable.
    \State Multiply factors and marginalize out the variable, keeping top \( k \) assignments.
    \State Update the remaining factors.
\EndFor
\State Normalize the probabilities using \( Z \).
\State Return the top \( k \) assignments.
\end{algorithmic}
\end{algorithm}

\section{References}
\begin{itemize}
    \item GitHub copilot, for code autocomplete
    \item Lecture slides, for the algorithm
\end{itemize}

\end{document}
