\documentclass[a4paper]{article} 
\addtolength{\hoffset}{-2.25cm}
\addtolength{\textwidth}{4.5cm}
\addtolength{\voffset}{-3.25cm}
\addtolength{\textheight}{5cm}
\setlength{\parskip}{0pt}
\setlength{\parindent}{0in}

%----------------------------------------------------------------------------------------
%	PACKAGES AND OTHER DOCUMENT CONFIGURATIONS
%----------------------------------------------------------------------------------------

\usepackage{blindtext} % Package to generate dummy text
\usepackage{charter} % Use the Charter font
\usepackage[utf8]{inputenc} % Use UTF-8 encoding
\usepackage{microtype} % Slightly tweak font spacing for aesthetics
\usepackage[english]{babel} % Language hyphenation and typographical rules
\usepackage{amsthm, amsmath, amssymb} % Mathematical typesetting
\usepackage{float} % Improved interface for floating objects
\usepackage[final, colorlinks = true, 
            linkcolor = black, 
            citecolor = black]{hyperref} % For hyperlinks in the PDF
\usepackage{graphicx, multicol} % Enhanced support for graphics
\usepackage{xcolor} % Driver-independent color extensions
\usepackage{marvosym, wasysym} % More symbols
\usepackage{rotating} % Rotation tools
\usepackage{censor} % Facilities for controlling restricted text
\usepackage{listings, style/lstlisting} % Environment for non-formatted code, !uses style file!
\usepackage{pseudocode} % Environment for specifying algorithms in a natural way
\usepackage{style/avm} % Environment for f-structures, !uses style file!
\usepackage{booktabs} % Enhances quality of tables
\usepackage{tikz-qtree} % Easy tree drawing tool
\tikzset{every tree node/.style={align=center,anchor=north},
         level distance=2cm} % Configuration for q-trees
\usepackage{style/btree} % Configuration for b-trees and b+-trees, !uses style file!
\usepackage[backend=biber,style=numeric,
            sorting=nyt]{biblatex} % Complete reimplementation of bibliographic facilities
\addbibresource{ecl.bib}
\usepackage{csquotes} % Context sensitive quotation facilities
\usepackage[explicit]{titlesec}
\titleformat{\section}{\normalfont\Large\bfseries}{}{0em}{#1\ \thesection}
\renewcommand{\thesubsection}{\thesection.\alph{subsection}}
\usepackage[yyyymmdd]{datetime} % Uses YEAR-MONTH-DAY format for dates
\renewcommand{\dateseparator}{-} % Sets dateseparator to '-'
\usepackage{fancyhdr} % Headers and footers
\pagestyle{fancy} % All pages have headers and footers
\fancyhead{}\renewcommand{\headrulewidth}{0pt} % Blank out the default header
\fancyfoot[L]{} % Custom footer text
\fancyfoot[C]{} % Custom footer text
\fancyfoot[R]{\thepage} % Custom footer text
\newcommand{\note}[1]{\marginpar{\scriptsize \textcolor{red}{#1}}} % Enables comments in red on margin
% -------------------Added by meeeeeeeee,Prateek
\usepackage{xcolor}
\usepackage{subcaption}

% \definecolor{codegreen}{rgb}{0,0.6,0}
% \definecolor{codegray}{rgb}{0.5,0.5,0.5}
% \definecolor{codepurple}{rgb}{0.58,0,0.82}
% \definecolor{backcolour}{rgb}{0.95,0.95,0.92}

% \lstdefinestyle{mystyle}{
%     backgroundcolor=\color{backcolour},   
%     commentstyle=\color{codegreen},
%     keywordstyle=\color{magenta},
%     numberstyle=\tiny\color{codegray},
%     stringstyle=\color{codepurple},
%     basicstyle=\ttfamily\footnotesize,
%     breakatwhitespace=false,         
%     breaklines=true,                 
%     captionpos=b,                    
%     keepspaces=true,                 
%     numbers=left,                    
%     numbersep=5pt,                  
%     showspaces=false,                
%     showstringspaces=false,
%     showtabs=false,                  
%     tabsize=2
% }

% \lstset{style=mystyle}
\DeclareMathOperator*{\argmax}{arg\,max}
\DeclareMathOperator*{\argmin}{arg\,min}
%----------------------------------------------------------------------------------------


\begin{document}

%-------------------------------
%	TITLE SECTION
%-------------------------------

\fancyhead[C]{}
\hrule \medskip % Upper rule
\begin{minipage}{0.295\textwidth} 
\raggedright
\footnotesize
Tejas, Muskaan, Ojas \hfill\\   
22B0909, 22B1058, 22B0965\hfill\\
Advanced Machine Unlearning
\end{minipage}
\begin{minipage}{0.4\textwidth} 
\centering 
\large 
Learning with Hidden Variables\\ 
\normalsize 
Advanced Machine Learning (Spring 2024)\\ 
\end{minipage}
\begin{minipage}{0.295\textwidth} 
\raggedleft
\today\hfill\\
\end{minipage}
\medskip\hrule 
\bigskip

%-------------------------------
%	CONTENTS
%-------------------------------

Some tasks, such as training Hidden Markov Models for natural language processing, require training with data available (supervision) only over a subset of the variables and prediction of all variables with the same, used in the generation of sentences. \hfill \\

We consider learning in an unconditional setting first since the conditional setting is just a variant with extra constraints. Further, we train our model as a whole using entire sentences rather than one variable at a time.

\section{The Framework}
We wish to learn $\vec{\theta}$, the parameters of the graphical model $G$ with set of cliques $C$, in order to maximize \[
P_{\vec{\theta}, G}\left( y_1, y_2, \dots, y_n, z_1, z_2, \dots z_m | \vec{x} \right) = 
\frac{1}{Z_{\vec{\theta}}(\vec{x})} \exp \left( \sum_{C} F_{\vec{\theta}}\left(\vec{y_C}, \vec{z_C}, \vec{x}\right) \right)
\] 
Specifically, given the dataset $D = \{ (\vec{x_i}, \vec{y_i}) : i \in \{ 1 \dots N \}  \}$  \[
\vec{\theta}_{ML}\ =\ 
\arg\max_{\vec{\theta}}\left( \sum_{i=1}^{N} \log\ P_{\vec{\theta}} ( \vec{y_i} | \vec{x_i} ) \right) \ =\  
\arg\max_{\vec{\theta}} \left( \sum_{i = 1}^{N} \log \left( \sum_{\vec{z}} P_{\vec{\theta}, G} ( \vec{y_i}, \vec{z} | \vec{x_i} ) \right) \right)
\]
This optimization problem, which contains a summation inside a logarithm, is difficult to solve. We will apply approximation algorithms such as the Expected Maximization (EM) algorithm to train such networks successfully.

\subsection{Variational Approach: Adding New Auxiliary Variables}
If we have a variable $z$ that takes $k$ values, we show that \[
\log\left( \sum_{z = 1}^{k} g(y, z) \right)\ =\ \max_{\vec{q}} Q(\vec{q}, g)\ 
\] where $\vec{q} = \{q_1, q_2, \dots q_k\}$ are auxiliary variables such that $\sum_{z} q_z = 1$ and $q_z \geq 0\ \forall z$ and
\[
Q(\vec{q}, g) =\ \sum_{z} q_z \log g(y, z) - \sum_{z} q_z \log q_z 
\]
\textbf{NOTE: }$- \sum_{z} q_z \log q_z $ is the \textbf{entropy} over $q_z$.\\ \\
\textbf{Proof:} \\ \\
We introduce the Lagrangian parameter $\lambda$, and define $$L(Q, \lambda) = Q(\vec{q}, g) + \lambda \left( \sum_{z} (q_z) - 1\right) $$
The second term evaluates to zero, but it eases calculation in the next step, where we attempt to find the maxima by differentiating w.r.t. all $q_j$. Specifically, we solve \[
\frac{\partial L(Q, \lambda)}{\partial q_j} = \log g(y, j) - \log q_j - 1 + \lambda = 0
\] This gives us 
$$q_j* = \alpha\cdot g(y, j)$$
for some constant $\alpha$ which is common across all $j$ and dependent on $\lambda$.

Using the normalization constraint on $q_z$, we get 
$$q_j* = \frac{{g(y, j)}}{\sum_z g(y, z)}$$
Substituting this, \[
Q(\vec{q}*, g)\ =\ \sum_z  q_z \log \left( \frac{g(y, z)}{q_z} \right)\ =\  
\sum_z \left( \frac{g(y, z)}{\sum_z g(y, z)} \log \left( \sum_z g(y, z) \right) \right)\ =\ 
\log \left( \sum_z g(y, z) \right)
\]

\subsection{Applying Variational Approach}
We assumed $z$ to be an integer variable in the proof, but the same does not use this fact and does not depend on the kind of $z$; $z$ could take any values, but $\vec{q}$ has as many values in it as the number of possible values of $z$. \hfill \\

We now consider the $i^{th}$ term in the outer sum alone, $\vec{z}$ instead of $z$ and $P_{\vec{\theta}, G} (\vec{y_i}, \vec{z} | \vec{x_i})$ as $g(y, z)$; $q_z$ is replaced by $q_{i, \vec{z}} $. Using the same theorem, \[
\log \left( \sum_{\vec{z}} P_{\vec{\theta}, G} ( \vec{y_i}, \vec{z} | \vec{x_i} ) \right)\ =\ 
\max_{\vec{q_i}\ :\ q_{i, \vec{z}} \geq 0,\ \sum_{\vec{z}} q_{i, \vec{z}} = 1 } \left( 
\sum_{\vec{z}} q_{i, \vec{z}} \log P_{\vec{\theta}, G} (\vec{y_i}, \vec{z} | \vec{z_i})\ -\ 
\sum_{\vec{z}} q_{i, \vec{z}} \log q_{i, \vec{z}} \right)
\] 
This reduces our task to solving for $\theta_{ML}$, the optimal parameters: \[
\vec{\theta}_{ML}\ =\  \arg\max_{\vec{\theta}} \sum_{i = 1}^{N} \left( \max_{\vec{q_i}\ :\ q_{i, \vec{z}} \geq 0,\ \sum_{\vec{z}} q_{i, \vec{z}} = 1 } \left( 
\sum_{\vec{z}} q_{i, \vec{z}} \log P_{\vec{\theta}, G} (\vec{y_i}, \vec{z} | \vec{z_i})\ -\ 
\sum_{\vec{z}} q_{i, \vec{z}} \log q_{i, \vec{z}} \right) \right)
\]
\textbf{Change of Notation} \\ \\
We now work only with generative models, and everything that follows can be extended to other models too. \\ \\
Before we proceed to the methods for solving for $q_{i, \vec{z}}$, we would change our problem notation: 
\begin{itemize}
    \item Without loss of generality, that the earlier $\vec{x}$ is absent.
    \item We use just $\vec{x}$ to represent what was earlier $\vec{y}$ 
    \item We express $P_{\vec{\theta}, G}(\vec{y_i}, \vec{z} | \vec{x_i}) $ as $P_{\vec{\theta}}(\vec{x_i}, \vec{z}) = P_{\vec{\theta}} (\vec{x_i} | \vec{z}) \cdot P_{\vec{\theta}} (\vec{z}) $.
    \item Thus, our optimization problem can also be expressed as \[
    \vec{\theta}_{ML} = \arg\max_{\vec{\theta}} \sum_{i = 1}^{N} \max_{\vec{q_i}} \left( 
    \sum_{\vec{z}} q_{i, \vec{z}} \log P_{\vec{\theta}}(\vec{x_i} | \vec{z}) -
    \sum_{\vec{z}} q_{i, \vec{z}} \log \left(\frac{P_{\vec{\theta}} (\vec{z})}{q_{i, \vec{z}}}\right) \right)
    \]
\end{itemize}

\subsection{Methods of solving $q_{i, \vec{z}}$} \begin{enumerate}
    \item \textbf{Expectation Maximization (EM) algorithm}: we alternatively solve for $\vec{q_{i}}$ and $\vec{\theta}$ keeping the other fixed. This is not tractable for high-dimensional $\vec{z}$, though.
    
    \item \textbf{Variational Auto-Encoders (VAE) algorithm}: we use a neural network which outputs $\vec{q_i}$ given input $\vec{x_i}$. Note that the neural network outputs the same in the form of a distribution $q_{\phi}(\vec{z} | \vec{x_i} )$.

    \item \textbf{Denoising Diffusion Models}: we break up $P_{\vec{\theta}}(\vec{x_i} | \vec{z}) $ term into many smaller steps, assuming a fixed $q_{i, \vec{z}}$ for each step.

    \item \textbf{Normalizing Flows (NF)}: We assume a simple, invertible function to implement $P_{\vec{\theta}}(\vec{x_i | \vec{z}})$ in order to compute the optimum $q_{i, \vec{z}} $ in closed form. This is the odd one out, and it relies on a deep generative model itself. 
\end{enumerate}
The NF method stands out among these, for reasons that we will see shortly. 

\section{The EM Algorithm}
We repeatedly find the optimum $\vec{q_i}$, fixing $\vec{\theta}$ and vice-versa. At the $t^{th}$ iteration,

\textbf{The E-step}: We solve for the posterior distribution of hidden variable $\vec{q_{i}}$ or $q_{i, \vec{z}}$ keeping $\vec{\theta} = \vec{\theta}_t$ fixed. \[
q_{t, i, \vec{z}} = \frac{P(\vec{x_i}, \vec{z}\ |\ \vec{\theta}_t)}{\sum_{\vec{z}} P(\vec{x_i}, \vec{z}\ |\ \vec{\theta}_t)} = P(\vec{z}\ |\ \vec{x_i}, \vec{\theta}_t) 
\]

\textbf{The M-step}: We solve for $\vec{\theta}$ keeping fixed $\vec{q_i} = \vec{q_{t, i}}$. \[
\vec{\theta}_{t + 1} = \arg\max_{\vec{\theta}}\sum_{i = 1}^{N} \sum_{\vec{z}} 
q_{t, i, \vec{z}} \log P(\vec{x_i}, \vec{z} | \vec{\theta})
\] This is concave in $\vec{\theta}$ and can often be solved in closed form, via factorization of $\vec{q_{t, i}}$ over cliques in graphical models, or directly in closed form in HMMs.
\hfill \\

\textbf{The overall algorithm}: This behaves somewhat like the block coordinate ascent algorithm, trying to optimize $\vec{q_i}$ and $\vec{\theta}$ alternatively in the E and M steps, and it can be shown that the algorithm is guaranteed to converge.

\subsection{Takeaways and Limitations of EM algorithm} \begin{itemize}
    \item The algorithm is efficient when dealing with tractable graphical models such as HMMs or graphical models (where the distribution can be factorized) and a small number of dimensions for $\vec{z}$.

    \item However, this becomes intractable where $\vec{z}$ has a lot of dimensions, since the number of possible values of $\vec{z}$ is $k^d$ and $\vec{q_i}$ has that many values. This cannot be used to train images, where each pixel constitutes three dimensions in the vector $\vec{z}$. This also cannot be used when $\vec{z}$ does not take discrete values.

    \item When not all training variables are observed in each training sample, learning of potentials that express interactions between variables becomes difficult.
\end{itemize}

\section{High Dimensional Objects and Continuous Domains}

Let $X \subset R^{n}$ be the space of all possible generated objects for a large $n$, and $C$ be the space of possible contexts corresponding to objects in $X$. We let $Z \subseteq R^{k_0}$ be the space of latent variables, which we assume to fit into some known distribution $P(Z)$, such as the standard Gaussian distribution. Unlike in the EM algorithm, we deal with continuous variables here. \hfill \\

Further, our algorithm depends on the existence of a parametric network $F_{\vec{\theta}}: Z, C \rightarrow X $, which is a high-capacity, deep neural network. \hfill \\

Our objective is to \begin{enumerate}
    \item Train the model $F_{\vec{\theta}} $: learn its parameters $\vec{\theta}$ using the dataset $D = \{ ( \vec{c}_i, \vec{x}_i )\}$ which contains $N$ objects, drawn from an unknown distribution $P(\vec{c}, \vec{x})$.
    \item Generate, given a context $\vec{c}$, a sentence $\vec{x} = F_{\vec{\theta}} (\vec{z}, \vec{c})$ where $z$ is sampled from $P(Z)$. Here, $\vec{x}$ may be a high-dimensional object that needs to be generated each time.
\end{enumerate}

Since the context $\vec{c}$ can be included later, we keep things simple, i.e., drop the context and just focus on unconditional generation.

\section{Variational Auto-Encoders}
Train $F_{\vec{\theta}}$ to model $P_{\vec{\theta}}(\vec{x} | \vec{z}) $. \\ \\
We have prior knowledge of $P(\vec{z})$, and $\vec{\theta}$ are the model parameters. \\
We need to learn $\vec{\theta}$s to maximize the likelihood of the training data. \\
The BN for this is:
$$
    \vec{z} \longrightarrow \vec{x}
$$

\subsection{What $P(\vec{x} | \vec{z})$ looks like?}
The distribution can be very complicated, for tasks like image generation etc. So, it is assumed to be a conditional Gaussian distribution, with parameters depending on $\vec{z}$ and on $\vec{\theta}$, which can be obtained when training the \textbf{decoder} neural network $F_{\vec{\theta}}$.
\[
P(\vec{x} | \vec{z}) = \mathcal{N}\left( \vec{\mu} = [ \mu_{x_1|\vec{z}}, \mu_{x_2|\vec{z}}, \dots \mu_{x_n|\vec{z}} ], \quad \mathbf{\Sigma} =  diag[\sigma_{x_1|\vec{z}}, \sigma_{x_2|\vec{z}}, \dots \sigma_{x_n|\vec{z}}] \right)
\]
Each $\mu_{x_i|\vec{z}}$ and $\sigma_{x_i|\vec{z}}$ depends on some of the parameters in $\vec{\theta}$.

\subsection{Training VAEs}
We are given training data:
\(
    D = \left\{\vec{x_i}\right\}_{i=1}^N
\)
and $\vec{z}$ is hidden.

We wish to compute 
\[
    \vec{\theta}_{ML}\ =\ \arg\max_{\vec{\theta}}\sum_{i=1}^{N} \log P_{\vec{\theta}}(\vec{x_i})
\]
Marginalizing out z to get to get $P_{\vec{\theta}}(\vec{x_i})$
\[
\arg\max_{\vec{\theta}}\sum_{i=1}^{N} \log P_{\vec{\theta}}(\vec{x_i})\ =\ 
\arg\max_{\vec{\theta}}\sum_{i=1}^{N} \log \int_{\vec{z}} P_{\vec{\theta}}(\vec{x} | \vec{z}) P_{\vec{\theta}}(\vec{z}) d\vec{z}
\] 
Here, $z \in \mathbb{R}^k$ and $x \in \mathbb{R}^n$ \\ \\
Therefore, integrating over all hidden variables $\vec{z}$ is intractable. \\ \\
So, we use the ``variational approximation'' (similar to the variational approach), i.e. 
\[
\log \int_{\vec{z}} P_{\vec{\theta}}(\vec{x}, \vec{z})d\vec{z}\ =\ \max_{q_{i,\vec{z}}} \int_{\vec{z}} \left( q_{i,\vec{z}} \log (P_{\vec{\theta}}(\vec{x}, \vec{z}).P(\vec{z})) - q_{i,\vec{z}} \log q_{i,\vec{z}} \right) d\vec{z}
\] 
where $q_{\vec{z}} \geq 0,\ \int_{\vec{z}}q_{\vec{z}}d\vec{z} = 1$ \\ \\
We cannot compute the integral for all $i,\vec{z}$,  so we have another intractable inner maximization problem. \\ \\
To get $q_{i, \vec{z}}$, we use another neural network:  $q_{\vec{\phi}}(\vec{z} | \vec{x_i}) $ with parameters $\vec{\phi}$ shared across all $i$. \\ \\
Thus, 
\[
\max_{\vec{\theta}}\sum_{i=1}^{N} \log P_{\vec{\theta}}(\vec{x_i}) \geq \max_{\vec{\theta}, \vec{\phi}} \sum_{i=1}^{N} \left( \underset{\text{First term}}
{\underbrace{\int_{\vec{z}} q_{\vec{\phi}}(\vec{z} | \vec{x_i}) \log \left( P_{\vec{\theta}}(\vec{x_i} | \vec{z}) \right) d\vec{z}}}\ - \underset{\text{Second term}}{\underbrace{
\int_{\vec{z}} q_{\vec{\phi}}(\vec{z} | \vec{x_i}) \log \left(\frac{q_{\vec{\phi}}(\vec{z} | \vec{x_i})}{P(\vec{z})} \right)d\vec{z}}} \right)
\]
We have taken out the factor $P(\vec{z})$ as it is independent of $\theta, \phi$, and doesn't affect the maximization problem.

\subsubsection{Estimating the first term}
We use sampling to estimate the first term.
Since 
$$
\int_{\vec{z}} q_{\vec{\phi}}(\vec{z} | \vec{x_i}) d\vec{z}\ =\ 1 
\quad \text{and} \quad
\forall \vec{z} \ \ q_{\vec{\phi}}(\vec{z} | \vec{x_i}) \geq 0
$$
We can say that the first term resembles a probability distribution of the variable $\vec{z}$ given $\vec{x_i}$. \\ \\
Using the expected value of a function of the variable
$$E_{X}[f(X)] = \int_{domain(X)} f(x) dx $$ 
the first term simplifies to 
$$
    E_{q_{\vec{\phi}} (\vec{z} | \vec{x_i})}\left[\log P_{\vec{\theta}} (\vec{x_i} | \vec{z_j}) \right]
$$
We now \textbf{approximate} this expectation as the average of $P_{\vec{\theta}}(\vec{x_i} | \vec{z}) $ over $r$ different samples of $\vec{z} \sim q_{\vec{\phi}}(\vec{z} | \vec{x_i})$: \[
E_{q_{\vec{\phi}} (\vec{z} | \vec{x_i})}\left[\log P_{\vec{\theta}} (\vec{x_i} | \vec{z}) \right]\ =\ 
\frac{1}{r}\sum_{j=1}^{r} \log P_{\vec{\theta}}(\vec{x_i} | \vec{z})
\]
How do we sample $\vec{z}$ from $q_{\vec{\phi}}$? \\ \\
\textbf{Assume}:  
\[
    q_{\vec{\phi}}(\vec{z} | \vec{x_i}) = \prod_{k} q_{\vec{\phi}}(z_k | \vec{x_i}) \quad \quad \text{where} \quad q_{\vec{\phi}}(z_k | \vec{x_i}) \sim \mathcal{N}\left(\mu_{z_k|\vec{x_i}}, \sigma^2_{z_k|\vec{x_i}} \right)
\] 
In order to get the desired $r$ samples of $\vec{z} \sim q_{\vec{\phi}}$ from this (multivariable) Gaussian distribution, we perform \textbf{re-parameterization}: Sample $v_{jk} \sim \mathcal{N}(0, 1)$ from the standard parameterless Gaussian distribution (mean 0, variance 1). Then, we express 
\[
    z_{jk} = \mu_{z_k |\vec{x_i}} + v_{jk}\sigma_{z_k |\vec{x_i}}\ \implies\ 
    \vec{z_j} = \vec{\mu}_{\vec{z} | \vec{x_i}} + \vec{v_j}\mathbf{\Sigma}_{\vec{z} | \vec{x_i}} \ \implies\
    \vec{z_j} \in \mathcal{N}\left(\vec{\mu}_{\vec{z}|\vec{x_i}}, \mathbf{\Sigma}_{\vec{z}|\vec{x_i}}\right) 
\]
Therefore, the first term is simplified to
\[
    \sum_{i=1}^N \frac{1}{r} \sum_{j=1;\ v_{jk} \sim \mathcal{N}(0,1)}^r \log P_{\theta}\left(
    \vec{x_i}\ |\ \vec{\mu}_{\vec{z} | \vec{x_i}} + \vec{v_j}\mathbf{\Sigma}_{\vec{z} | \vec{x_i}} \right)
\]
For this purpose, we assume an `encoder' neural network $q_{\vec{\phi}}$, which accepts $\vec{x}$ as input and outputs $\vec{\mu}$ and $\mathbf{\Sigma}$. 

\subsubsection{Simplifying the second term}
The second term (integral) is the KL distance between $q_{\vec{\phi}}(\vec{z} | \vec{x_i}) $ and $P_{\vec{\theta}}(\vec{z}) $. \\ \\
Under the (already known) assumptions 
$$P(\vec{z}) \sim \mathcal{N}(0, 1)$$
$$q_{\vec{\phi}}(\vec{z} | \vec{x_i}) \sim \mathcal{N}\left(\vec{\mu_i}(\vec{x}), \mathbf{\Sigma_i}(\vec{x})\right) $$ 
We have, 
\[
\int_{\vec{z}}\left(q_{\vec{\phi}}(\vec{z} | \vec{x_i})\log \frac{q_{\vec{\phi}}(\vec{z} | \vec{x_i})}{P_{\vec{\theta}}(\vec{z})} \right) d\vec{z}\ =\ 
\int_{\vec{z}}  \mathcal{N}\left(\vec{\mu_i}, \mathbf{\Sigma_i}\right) \cdot \frac{1}{2} \left( - \frac{\left\| \vec{z} - \vec{\mu_i} \right\|_2^2}{\left| \mathbf{\Sigma_i} \right| } + \left\|\vec{z}\right\|_2^2 - \log \left| \mathbf{\Sigma_i} \right|  \right) d \vec{z}
\] 
This evaluates to 
$$
    \frac{1}{2}\left( \left\| \vec{\mu_i} \right\|_2^2 + \left| \mathbf{\Sigma_i} \right| - 1 - \log  \left| \mathbf{\Sigma_i} \right| \right)
$$

\subsubsection{Putting it all together} 
\[
    \vec{\theta}_{ML}, \vec{\phi}_{ML} = \arg\max_{\vec{\theta}, \vec{\phi}} \sum_{i=1}^{n} \left(
    \frac{1}{r}\sum_{j=1}^{r} \log P_{\vec{\theta}}\underset{ \vec{v_j}\ :\ v_{j, k} \sim N(0, 1)}
    {\underbrace{ \left(\vec{x_i}\ |\ \vec{\mu}(\vec{x_i}) + \vec{v_j}\mathbf{\Sigma}(\vec{x_i}) \right)}} \ -\ 
    \frac{1}{2}\left( \left\| \vec{\mu}(\vec{x_i}) \right\|_2^2 + \left| \mathbf{\Sigma}(\vec{x_i}) \right| - 1 - \log  \left| \mathbf{\Sigma} (\vec{x_i}) \right| \right)
    \right)
\]  \\ 
where $P_{\vec{\theta}}(\vec{x} | \vec{z}) = \mathcal{N}\left(\vec{\mu}(\vec{z}), \mathbf{\Sigma}(\vec{z})\right)(\vec{x})$. \\
Here, $\vec{\mu}(\vec{z}), \mathbf{\Sigma}(\vec{z})$ are  generated from the `decoder' network $F_{\vec{\theta}}$, and $\vec{\mu}(\vec{x}), \mathbf{\Sigma}(\vec{x})$ are generated from the `encoder' network $q_{\vec{\phi}} $.

\subsection{Training the Model: The Algorithm}
We train $\vec{\theta}$ and $\vec{\phi}$ for the two networks. The overall algorithm uses batch gradient descent. \hfill \\

We start by initializing them \textbf{randomly}. \\ \\
We train for a set number of iterations $T$. In each iteration $t$, we take a \textbf{mini-batch} of $B$ data points. Without loss of generality, let the data points be $\vec{x_1}, \vec{x_2}, \dots \vec{x_B}$.

In each iteration: 
\begin{enumerate}
    \item For $i \in \{1, 2, \dots B\}$, compute $\vec{\mu}(\vec{x_i})$ and $\mathbf{\Sigma}(\vec{x_i})$ using the encoder $q_{\vec{\phi}}$.
    \item Get $Br$ samples of $\vec{v}$ : $v_{i, j, k} \sim N(0, 1)$, and $\vec{v_{i, j}}$ is a $k_0$-dimensional vector.
    \item Compute $\vec{z_{i, j}} = \vec{\mu}(\vec{x_i}) + \vec{v_{i, j}}\mathbf{\Sigma}(\vec{x_i}) $ for all $Br$ values of $(i, j)$.
    \item For all $Br$ values of $(i, j)$ compute $\vec{\mu}(\vec{z_{i, j}})$ and $\mathbf{\Sigma}(\vec{z_{i, j}})$ from the decoder $F_{\vec{\theta}}$.
    \item Compute the loss function: 
    \[
        loss\left(\vec{\theta}, \vec{\phi}\right) = \sum_{i=1}^{B}\left( 
        - \frac{1}{r}\sum_{j=1}^{r} \Big(\log N\big( \vec{\mu}(\vec{z_i}), \mathbf{\Sigma}(\vec{z_{i,j}}) \big)(\vec{z_{i,j}})\Big) + \left\| \vec{\mu}(\vec{x_i}) \right\|^2 + \left| \mathbf{\Sigma}(\vec{x_i}) \right| - \log \left| \mathbf{\Sigma}(\vec{x_i}) \right|
        \right)
    \]
    \item Perform the gradient descent update using the derivative of the loss with respect to $\vec{\theta}$ and $\vec{\phi}$
\end{enumerate}
\subsection{Limitation of VAE}
Training $\vec{\theta}$ and $\vec{\phi}$ for the encoder and decoder separately is difficult, and the resultant quality is poor.

\end{document}

